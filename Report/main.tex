%%% Compile using: pdflatex main.tex %%%
\documentclass[a4paper,12pt]{article}
\usepackage{mypackages}

\title{\textbf{MNXB01} - Project report}
\date{\today}
\author{Cameron Robertson, \and Viktor Drugge, \and Louise Villander}

\begin{document}
	\maketitle
	\section{Introduction}
	\label{sec:pre}
	In this report, urban adjusted temperatures from the Uppsala data center will be analysed
	and graphed, with a particular focus on Spring temperatures. Samples of temperatures during certain
	days will also be plotted. Finally, the data will be used to determine the beginning of Spring, through
	the temperature definition. To do this, ROOT version 5.34/30 is used.
	\documentclass[a4paper,10pt,oneside]{article}

%\usepackage{ucs}
%\usepackage[utf8]{inputenc}
%\usepackage{babel}
%\usepackage{fontenc}
%\usepackage[pdftex]{graphicx}

\usepackage{mypackages}

\author{Cameron Robertson}
\date{11/09/17}

\begin{document}

\section{Introduction}
\label{sec:intro}

 In this section, the temperature
data from Uppsala will be analysed with the aid of histograms, which show the mean daily
temperatures from 1722 until 2013. Each histogram will show data for a specific day of the year,
so it may be enlightening to look at all 366 potential histograms, though this would
not be practical. Instead, a few specific dates will be chosen to analyse the histogram
distribution. It will then be determined how useful these results will be in determining the beginning
of Spring, and also how much temperatures have changed in the last three hundred years.

\section{Extracting the data}
\label{sec:data}

The Uppsala temperature data was given as a space seperated list, with the first three
columns representing the year, month and day respectively. The fourth column included the mean
temperature of that day unadjusted for urban effect, whereas the fifth column held
the same, but adjusted for urban effect. The temperatures to be plotted were those 
in the fifth column. The sixth held an id number to represent the weather station
that the temperature was recorded from. For the purposes of this project, only Uppsala
data was needed, which had an id of 1.

The function to extract the required information took two arguments: a day and a month. Firstly, each column was streamed into a vector. This was
easier and safer than using an array, since vectors dynamically change size. A for loop was then run which streamed the appropriate vector elements for the
chosen month and day into another file. This also only streamed the data with an id of 1. The data in the file was then streamed back into another
vector, which was then used to plot the histogram

\section{Results and Discussion}
\label{sec:res}

Results from the tempreal(inday,inmonth) function follow

\begin{figure}[!ht]
  \begin{center}
    \includegraphics[width=5cm]{../Code/Equinox_temps.jpg}
    \caption{Histogram showing temperatures from 1722-2013 at 21.3}
    \label{fig:vern}
  \end{center}
\end{figure}

Clearly \ref{fig:vern} shows that very generally, the temperature in Uppsala at the
vernal equinox tends to be below 0 degrees, which is colder than previously expected, especially given the previous
results from this report. However, the rather high standard deviation
shown in \ref{fig:vern} displays just how much variation in temperature there is over 300 years.
The range is also expectedly quite large, owing to the large quantity of data. Also, there have been several cases of extremely
cold temperatures, which could have altered the mean.

\begin{figure}[!ht]
 \includegraphics[width=5cm]{../Code/week.jpg}
 \caption{Histogram showing temperatures from 1722-2013 at 30.3}
 \label{fig:week}
\end{figure}

Comparing \ref{fig:week} to \ref{fig:vern}, the mean temperature has clearly increased significantly even
after just one week, with the mean temperature now being above freezing. Notably, the standard deviation
has also decreased slightly, so the spread of results for this day is slightly lower than on the previous histogram, though is
still quite high.

\begin{figure}[!ht]
 \includegraphics[width=5cm]{../Code/2week.jpg}
 \caption{Histogram showing temperatures from 1722-2013 at 6.3}
 \label{fig:twoweek}
\end{figure}

The same trend continues one week on from this, as shown by \ref{fig:twoweek}. The mean temperature has again increased significantly
, though by a smaller value this time. Most notably, the spread of data has decreased, resulting in a visibly thinner histogram, resulting
in a lower standard deviation.

\section{Results for Progression}
\label{sec:prog}

There has clearly been quite some variation of the mean temperature on a certain day within
this time period. By seperating the data into three vectors and plotting these seperately, an increase
in mean temperature can be seen in the last hundred years.
\begin{figure}[!ht]
 \includegraphics[width=5cm]{../Code/1.jpg}
 \caption{Histogram showing temperatures from 1722 to approx 1820, on 21.3}
 \label{fig:one}
\end{figure}

\begin{figure}[!ht]
 \includegraphics[width=5cm]{../Code/2.jpg}
 \caption{Histogram showing temperatures from approx 1820 to approx 1920, on 21.3}
 \label{fig:two}
\end{figure}

\begin{figure}[!ht]
 \includegraphics[width=5cm]{../Code/3.jpg}
 \caption{Histogram showing temperatures from approx 1920 to 2013, on 21.3}
 \label{fig:three}
\end{figure}

While the mean temperature is approximately the same between the first two time periods
shown, there is a clear increase in mean temperature in the last hundred years. Whilst the graphs
shown are just for 21st March, this trend is seen on all days tested.


\section{Conclusion}
\label{sec:conc}

The results show that the large amount of data nevertheless still shows the expected trend of increasing temperature through a year from week
to week, despite large standard deviation and recurring data from very cold years, which can be seen on
the left side of all the histograms. Theoretically, it would be possible to find the mean warmest and coldest day between 1722-2013,
but this would require repeatedly running the aforementioned function until the minimum and maximum values were found
It would also be difficult to determine the beginning of Spring(Using the temperature definition) using this program,
as a similar brute force method would be required, and it would be difficult to plot the results. Furthermore, seperating
the data also allowed a clear increase in mean temperature to be seen in the last hundred years.






 
\end{document}

	\documentclass[a4paper,12pt]{article}
\usepackage{mypackages}

\begin{document}
\section{Calculating the start of spring}

\end{document}

	\documentclass[a4paper,12pt,twoside]{article}
\usepackage{mypackage}

\begin{document}
 \section{Calculating the mean temperatures of every day of the year}
 \label{sec:3.2}
 How does the temperature change over the course of a year? Recording the temperature
 every day for one year might give some indication, but leaves a large margin of error.
 With a dataset stretching almost three hundred years we can calculate the mean
 temperature of each day and get a much more reliable result.
 
 The mean temperature of a certain day is the sum of the temperatures recorded at
 that day, divided by the number of days it has been recorded:
 \begin{equation}
  \label{eq:MeanTempPerDay}
  \overline{T(day 1)} = \frac{T(day 1, year 1)+T(day 1, year 2)+...+T(day 1, year N)}{N}
 \end{equation} 
  
 
 Reading from the datafile, each datatype was put in a vector (excluding the year
 1722 since we don't have all the data for that year, and the 29th of february for
 simplicity). We now have the data in vectors of equal length $(2013-1722)\times(365)=106215$,
 where the data in the same position in the vectors correlate.
 
 Next, the data needed to be separated according to which year it belonged. An empty
 vector of length 365 was created, and then (using a for-loop) the temperature belonging
 to day $k$ was added to the $(k-1)$:th position in the vector (excluding the data not from
 Uppsala. The number of times this summation happened was also recorded (it should be
 equal to the number of years, $291$).
 
 \begin{verbatim}
  	//loop for calculating the mean temperatures of each day
	for (UInt_t k=0; k < vyear.size(); k++){
	
	...
	
	//for a new year
	else if (vyear.at(k) != vyear.at(k+1)){
		daycounter = 365;
	}
	
	//for the same year,
	    the daycounter is that of the previous iteration + 1
	else if (vyear.at(k) == vyear.at(k+1)){
		daycounter = daycounter+1;
	}
	
	sumtempvec.at(daycounter-1)
	          =sumtempvec.at(daycounter-1) + vtemp.at(k);
	
	...
	
	if (daycounter == 365){
		daycounter = 0;
		yr_it = yr_it +1; //the number of years we've iterated over
	}
	}
 \end{verbatim}
 Using another for-loop to divide each element of the vector with the number of years,
 we now had a vector where each element was the mean temperature of the day
 corresponding to the element's position.
 
 \bigskip
 The standard deviation is defined as 
  \begin{equation}
 \label{eq:StdDev}
 \sigma = \sqrt{\frac{\sum\limits_{i=1}^{n} (t_{i}-\bar{t})^2}{n-1}}
 \end{equation}
 
 The mean temperature of each day $\bar{t}$ had been calculated as above, and now another 
 empty vector of length 365 was created. In the same way as before, we looped over the vectors
 containing the data, this time putting the square of the sum of the differences between the
 temperature of the current iteration and the mean temperature of the current day in the vector:
 
 \begin{verbatim}
    	...
  	//put the difference between the temperature and the mean
  	        temperature squared in a vector
  	vStdDev.at(daycounter-1) +=
  	    (vtemp.at(k)-sumtempvec.at(daycounter-1))*
  	    (vtemp.at(k)-sumtempvec.at(daycounter-1));
  	
  	...
  	
  	//calculating the standard deviation for each day
	for (UInt_t q=0;q<vStdDev.size();q++){
		vStdDev.at(q) = TMath::Sqrt(vStdDev.at(q)/(yr_it-1));
	}
 \end{verbatim}

 
 After the iteration we therefore had ${\sum\limits_{i=1}^{n} (t_i - \bar{t})}^2$. Using another for-loop, each element in the vector
 was then divided by the number of years ($n$) in the same way as before, and the square root of these values was
 calculated, to get the standard deviation for each day of the year.
 
 \begin{verbatim}
  	//calculating the standard deviation for each day
	for (UInt_t q=0;q<vStdDev.size();q++){
		vStdDev.at(q) = TMath::Sqrt(vStdDev.at(q)/(yr_it-1));
	}
 \end{verbatim}
 Finally we had two vectos, one containing the mean temperatures of each day, and the other
 containing the standard deviation, where the position in the vector correlated to the day
 of a year. Using yet another for-loop, the elements in the two vectors were set to be the
 bin content and the bin errors in a histogram. This produced the histogram shown in
 figure \ref{fig:tempPerDay}.
 
 \begin{figure}[h]
  \begin{center}
   \includegraphics[width=15cm]{../Code/tempPerDay.jpg}
   \caption{Histogram showing the mean temperature of every day of the year}
   \label{fig:tempPerDay}
  \end{center}
 \end{figure}
 
 
\end{document}




	
	\section{Conclusion}
	\label{sec:con}
	This report has demonstrated some of the results from analysis of the Uppsala dataset. Despite the sporadic nature of temperature when 
	taken over a long period of time, we have seen that viable conclusions and results can be derived. The mean temperature of a given day was seen to be quite variable depending on 
	which day was inputted, but the general trend over a number of days was as expected (increase throughout late March). Interestingly, extracting the spring date of each year
	suggested March 20th or March 21th as the average date of spring arrival, in line with the commonly accepted date. Plotting the temperature of all determined dates resulted in a 
	histogram that fitted well with a exponential function, this is as expected since on average the temperature increase gradually each spring. Leap years shift dates after the 28th
	of february which affect the results, but this is of minor concern since the standard deviation is relatively high, even when shifting the dates by one day. Plotting the mean temperatures
	of every day shows how much temperatures vary over the course of a year, and also that temperatures can vary quite a lot from year to year on the same date, particularly during the end 
	of winter and in spring.
\end{document}
